\documentclass[11pt,a4paper,dvipdfmx]{article}
%\documentclass[autodetect-engine,dvipdfmx-if-dvi,ja=standard]{bxjsarticle}

\usepackage[utf8]{inputenc}
\usepackage{lmodern}
\usepackage[T1]{fontenc}
\usepackage[noBBpl]{mathpazo}
%\linespread{1.05}
\usepackage{mathtools, amsmath, amssymb, amsthm}
\usepackage{amsfonts}
\usepackage{braket}
%\usepackage{amssymb}
\usepackage{url}
\usepackage{cases}
\usepackage{bbm}

%% citation
\usepackage[longnamesfirst]{natbib}

%
\theoremstyle{plain}
\newtheorem{thm}{Thm.}[section]
\newtheorem{lem}{Lem.}[section]
\newtheorem{cor}{Cor.}[section]
\newtheorem{prop}{Prop.}[section]
\newtheorem{df}{Def.}[section]
\newtheorem{eg}{e.g.}[section]
\newtheorem{rem}{Rem.}[section]
%

\usepackage{listings}
\lstset{%
language={python},%
basicstyle={\ttfamily\footnotesize},%ソースコードの文字を小さくする
frame={single},
commentstyle={\footnotesize\itshape},%コメントアウトの文字を小さくする
breaklines=true,%行が長くなったときの改行。trueの場合は改行する。
numbers=left,%行番号を左に書く。消す場合はnone。
xrightmargin=3zw,%左の空白の大きさ
xleftmargin=3zw,%右の空白の大きさ
stepnumber=1,%行番号を1から始める場合こうする(たぶん)
numbersep=1zw,%行番号と本文の間隔。
}

%\usepackage[dvipdfmx]{graphicx}
%% color packageとdvipdfmxは相性が悪いらしい
%% https://qiita.com/zr_tex8r/items/442b75b452b11bee8049
\usepackage{graphicx}


\usepackage[left=2cm,right=2cm,top=2cm,bottom=2cm]{geometry} %This changes the margins.
\usepackage{float}
%\author{Kyohei Okumura}
\global\long\def\T#1{#1^{\top}}

\newcommand{\id}{\textnormal{id}}
\newcommand{\R}{\mathbb{R}}
\newcommand{\N}{\mathbb{N}}
\newcommand{\Q}{\mathbb{Q}}
\newcommand{\Z}{\mathbb{Z}}
\newcommand{\C}{\mathbb{C}}
\newcommand{\mF}{\mathcal{F}}
\newcommand{\mG}{\mathcal{G}}
\newcommand{\mA}{\mathcal{A}}
\newcommand{\mB}{\mathcal{B}}
\newcommand{\mC}{\mathcal{C}}
\newcommand{\mD}{\mathcal{D}}
\newcommand{\mL}{\mathcal{L}}
\newcommand{\mN}{\mathcal{N}}
\newcommand{\mM}{\mathcal{M}}
\newcommand{\mO}{\mathcal{O}}
\newcommand{\mP}{\mathcal{P}}
\newcommand{\mS}{\mathcal{S}}
\newcommand{\mT}{\mathcal{T}}
\newcommand{\mV}{\mathcal{V}}
\renewcommand{\Re}{\mathrm{Re}}
\renewcommand{\hat}{\widehat}
\renewcommand{\tilde}{\widetilde}
\renewcommand{\bar}{\overline}
\renewcommand{\epsilon}{\varepsilon}
% \renewcommand{\span}{\mathrm{span}}
\newcommand{\defi}{\stackrel{\Delta}{\Longleftrightarrow}}
\newcommand{\equi}{\Longleftrightarrow}
\newcommand{\s}{\succsim}
\newcommand{\p}{\precsim}
\newcommand{\join}{\vee}
\newcommand{\meet}{\wedge}
\newcommand{\E}{\mathbbm{E}}
\newcommand{\1}{\mathbbm{1}}

\DeclareMathOperator{\Var}{Var}
\DeclareMathOperator{\Cov}{Cov}
\DeclareMathOperator{\sgn}{sgn}
\DeclareMathOperator{\Card}{Card}
\DeclareMathOperator{\supp}{supp}
\DeclareMathOperator{\Log}{Log}
\DeclareMathOperator{\spn}{span}
\DeclareMathOperator*{\argmin}{argmin}
\DeclareMathOperator*{\argmax}{argmax}

\newcommand{\indep}{\mathop{\perp\!\!\!\!\perp}}

\usepackage{color}
\newcommand{\kcomment}[1]{{\textcolor{blue}{#1}}}
\newcommand{\ocomment}[1]{{\textcolor{red}{#1}}}


\begin{document}
\title{Notes on Mechanism Design}
\author{Kyohei OKUMURA
%{\footnote{E-mail: kyohei.okumura@gmail.com}
%\footnote{UTokyo Econ M2}}
}
\date{\today}
\maketitle

%%%%%%%%%%%%%%%%%%%%%%%%%%%%%%%%%%%%%%%%%%%%%%%%%%%%%%%%%%%
\begin{itemize}
	\item This study notes are mainly based on the lecture note written by Valimaki in 2018.
\end{itemize}

\section{Single Agent}
\begin{itemize}
	\item One principal v.s. one agent.
	\item $a \in A$: allocation, $\theta \in \Theta$: agent's private info. $\theta \sim F(\theta)$.
	\item $u^P(a, \theta)$, $u^A(a, \theta)$
	\item We often assume quasi-linear payoff functions:
	\item $a := (x, t)$, $u^P(a, \theta) := v^P(x, \theta) + t$, $u^A(a, \theta) := v^A(x, \theta) - t$.
	\item As for implementability, we can discuss it focusing only on direct mechanisms, $(\Theta, \phi)$, w.l.o.g. (Revelation principle)
\end{itemize}

\subsection{Revenue Equivalence}
\subsubsection{Milgrom and Segal (2002), Envelope Theorem}
\begin{itemize}
	\item $\Theta := [\underline{\theta}, \bar{\theta}]$. $f(\cdot, \theta): X \to \R$. $\{f(\cdot, \theta)\}_{\theta \in \Theta}$.
	\item $V(\theta) := \max_{x \in X} f(x, \theta)$. $X^*(\theta) := \argmax_{x \in X} f(x, \theta)$
\end{itemize}
\begin{df}[Selection]
	A function $x^*: \Theta \to X$ is a selection from $X^*$ if $x^*(\theta) \in X^*(\theta)$ for all $\theta \in \Theta$.
\end{df}
\begin{thm}[Milgrom and Segal (2002)]
	Assume the following:
	\begin{itemize}
		\item For any $x \in X$, $f(x, \cdot): \Theta \to \R$ is absolutely continuous on $\Theta$.
		\item For any $x \in X$, $f(x, \cdot): \Theta \to \R$ is differentiable on $\Theta$.
	\end{itemize}
	Then, the following holds:
	\begin{itemize}
		\item $V$ is absolutely continuous.
		\item For any selection $x^*$ from $X^*$, $V(\theta) = V(\underline{\theta}) + \int_{\underline{\theta}}^\theta f_\theta(x^*(s), s) ds$.
	\end{itemize}
\end{thm}
\begin{proof}
	Note that the absolute continuity of $f(x, \theta)$ implies that $f_\theta(x, \theta) \in L^1(\Theta)$ for any $x \in X$.
	\paragraph{(i) $V$ is absolutely continuous.}
	It is sufficient to show that $V$ is Lipschitz continuous. Fix any $\theta', \theta$. Since any integrable function is bounded, for any $x$ there exists $L>0$ s.t. $|f_\theta(x, \theta)| \leq L$ for almost all $\theta \in \Theta$.
	\begin{align*}
		|V(\theta') - V(\theta)|
		&=
		\left|
		\max_{x'} f(x', \theta') - \max_x f(x, \theta)
		\right| \\
		&\leq
		\max_x
		\left|
		f(x, \theta') - f(x, \theta)
		\right| 
		=
		\max_x
		\left|
		\int_{\theta'}^\theta f_\theta(x, s) ds
		\right| \\
		&\leq L \cdot |\theta' - \theta|
	\end{align*}
	
	\paragraph{(ii)}
	Fix any selection $x^*$ from $X^*$. By the result of (i),
	\[
	V(\theta) = V(\underline{\theta}) + \int_{\underline{\theta}}^\theta V'(s) ds
	\]
	Fix any selection $x^*$ and $\theta', \theta$ such that $\theta' > \theta$. By the definition of $V$ and $x^*$,
	\begin{align*}
		V(\theta) &= f(x^*(\theta), \theta) \geq f(x^*(\theta'), \theta) \\
		V(\theta') &= f(x^*(\theta'), \theta') \geq f(x^*(\theta), \theta')
	\end{align*}
	Hence,
	\[
	V(\theta') - V(\theta) \leq f(x^*(\theta'), \theta') - f(x^*(\theta'), \theta).
	\]
	\[
	\frac{V(\theta') - V(\theta)}{\theta' - \theta} \leq \frac{f(x^*(\theta'), \theta') - f(x^*(\theta'), \theta)}{\theta' - \theta}.
	\]
	Similarly,
	\[
	V(\theta) - V(\theta') \leq f(x^*(\theta'), \theta) - f(x^*(\theta'), \theta').
	\]
	\[
	\frac{V(\theta') - V(\theta)}{\theta - \theta'} \geq \frac{f(x^*(\theta'), \theta) - f(x^*(\theta'), \theta')}{\theta - \theta'}.
	\]
	Note that by assumption $f(x, \cdot)$ is differentiable at all $\theta \in \Theta$. Therefore, if $V$ is differentiable at $\theta$, we have $V'(\theta) = f_\theta(x^*(\theta), \theta)$.
\end{proof}

\subsubsection{RET}
\begin{itemize}
	\item Focus on the agent's utility: $u := u^A$.
	\item $A := \phi(\Theta)$. $V(\theta) := \max_{a \in A} u(a, \theta)$. $A^*(\theta) := \argmax_{a \in A} u(a, \theta)$.
	\item Assume that $u(a, \cdot)$ is absolutely continuous and differentiable on $\Theta$ for all $a \in A$.
	\item By incentive compatibility, $\phi(\theta) \in A^*(\theta)$ for all $\theta \in \Theta$: $\phi$ is a selection from $A^*$.
\end{itemize}

\begin{thm}[Revenue Equivalence Theorem]
	\[
	V(\theta) = V(\underline{\theta}) + \int_{\underline{\theta}}^\theta
	 u_\theta(\phi(s), s) ds
	\]
	In particular, under quasi-linear utility, 
	\[
	V(\theta) = V(\underline{\theta}) + \int_{\underline{\theta}}^\theta
	 v_\theta(x(s), s) ds
	\]
	\[
	t(\theta) = v(x(\theta), \theta) - V(\underline{\theta}) - \int_{\underline{\theta}}^\theta
	 v_\theta(x(s), s) ds
	\]
\end{thm}
\begin{proof}
	Milgrom and Segal. As for quasi-linear cases, the results follow from
	$$
	V(\theta) = v(x(\theta), \theta) - t(\theta)
	$$
\end{proof}

\begin{itemize}
	\item RET states that under any IC mechanism, except for the constant $V(\underline{\theta})$, the transfer from the agent to the principal is uniquely determined once the allocation rule $x$ is fixed.
\end{itemize}

\subsection{Characterization of IC}
\subsubsection{Monotone Comparative Statics}
\begin{itemize}
	\item Consider parameterized optimization problem.
	\item We often want to know how optimizers and optimal values change according to the changes in parameters.
	\item Implicit function theorem: many assumptions are required.
	\item Monotone comparative analysis = Sensitivity analysis
	\item $\Theta \subseteq \R$. Two functions $g: \Theta \to \R$ and $f: \Theta \to \R$.
\end{itemize}

\begin{df}[Single Crossing]
	$g$ dominates $f$ by single crossing property (SCP), $g \succsim_{SC} f$, if for all $x'' > x'$,
	\begin{itemize}
		\item $f(x'') - f(x') \geq 0 \implies g(x'') - g(x') \geq 0$
		\item $f(x'') - f(x') > 0 \implies g(x'') - g(x') > 0$
	\end{itemize}
	
	$\{f(\cdot, \theta)\}_{\theta \in \Theta}$ is an SCP family if
	$$
	\forall \theta'' > \theta' ; \ f(\cdot, \theta'') \succsim_{SC} f(\cdot, \theta')
	$$
\end{df}

\begin{df}[Increasing Differences]
	$g$ dominates $f$ by increasing differences, $g \succsim_{IN} f$, if for all $x'' > x'$, $$g(x'') - g(x') \geq f(x'') - f(x').$$
	$\{f(\cdot, \theta)\}_{\theta \in \Theta}$ satisfies increasing differences if
	$$
	\forall \theta'' > \theta' ; \ f(\cdot, \theta'') \succsim_{IN} f(\cdot, \theta')
	$$
\end{df}

\begin{itemize}
	\item Note that $g \succsim_{IN} f$ implies $g \succsim_{SC} f$.
\end{itemize}

\begin{thm}[Milgrom and Shannon]
	$X \subseteq \R$. $f,g: X \to \R$.
	\[
	[\forall Y \subseteq X; \ \argmax_{x \in Y} g(x) \geq \argmax_{x \in Y} f(x)
	]
	\equi g \succsim_{SC} f
	\]
	Note that 
	\[
	\argmax_{x \in Y} g(x) \geq \argmax_{x \in Y} f(x)
	\defi  [z \in \argmax_{x \in Y} g(x), \ w \in \argmax_{x \in Y} f(x)
	\implies z \geq w
	]
	\]
\end{thm}
\begin{proof}.
	\paragraph{$\Rightarrow)$}
	Fix any $x'', x'$ such that $x'' > x'$.
\end{proof}

\subsubsection{Characterization of IC}


\end{document}