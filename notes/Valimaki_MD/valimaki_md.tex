\documentclass[11pt,a4paper,dvipdfmx]{article}
%\documentclass[autodetect-engine,dvipdfmx-if-dvi,ja=standard]{bxjsarticle}

\usepackage[utf8]{inputenc}
\usepackage{lmodern}
\usepackage[T1]{fontenc}
\usepackage[noBBpl]{mathpazo}
%\linespread{1.05}
\usepackage{mathtools, amsmath, amssymb, amsthm}
\usepackage{amsfonts}
\usepackage{braket}
%\usepackage{amssymb}
\usepackage{url}
\usepackage{cases}
\usepackage{bbm}

%% citation
\usepackage[longnamesfirst]{natbib}

%
\theoremstyle{plain}
\newtheorem{thm}{Thm.}[section]
\newtheorem{lem}{Lem.}[section]
\newtheorem{cor}{Cor.}[section]
\newtheorem{prop}{Prop.}[section]
\newtheorem{df}{Def.}[section]
\newtheorem{eg}{e.g.}[section]
\newtheorem{rem}{Rem.}[section]
\newtheorem{ass}{Ass.}
%

\usepackage{listings}
\lstset{%
language={python},%
basicstyle={\ttfamily\footnotesize},%ソースコードの文字を小さくする
frame={single},
commentstyle={\footnotesize\itshape},%コメントアウトの文字を小さくする
breaklines=true,%行が長くなったときの改行。trueの場合は改行する。
numbers=left,%行番号を左に書く。消す場合はnone。
xrightmargin=3zw,%左の空白の大きさ
xleftmargin=3zw,%右の空白の大きさ
stepnumber=1,%行番号を1から始める場合こうする(たぶん)
numbersep=1zw,%行番号と本文の間隔。
}

%\usepackage[dvipdfmx]{graphicx}
%% color packageとdvipdfmxは相性が悪いらしい
%% https://qiita.com/zr_tex8r/items/442b75b452b11bee8049
\usepackage{graphicx}


\usepackage[left=2cm,right=2cm,top=2cm,bottom=2cm]{geometry} %This changes the margins.
\usepackage{float}
%\author{Kyohei Okumura}
\global\long\def\T#1{#1^{\top}}

\newcommand{\id}{\textnormal{id}}
\newcommand{\R}{\mathbb{R}}
\newcommand{\N}{\mathbb{N}}
\newcommand{\Q}{\mathbb{Q}}
\newcommand{\Z}{\mathbb{Z}}
\newcommand{\C}{\mathbb{C}}
\newcommand{\mF}{\mathcal{F}}
\newcommand{\mG}{\mathcal{G}}
\newcommand{\mA}{\mathcal{A}}
\newcommand{\mB}{\mathcal{B}}
\newcommand{\mC}{\mathcal{C}}
\newcommand{\mD}{\mathcal{D}}
\newcommand{\mL}{\mathcal{L}}
\newcommand{\mN}{\mathcal{N}}
\newcommand{\mM}{\mathcal{M}}
\newcommand{\mO}{\mathcal{O}}
\newcommand{\mP}{\mathcal{P}}
\newcommand{\mS}{\mathcal{S}}
\newcommand{\mT}{\mathcal{T}}
\newcommand{\mV}{\mathcal{V}}
\renewcommand{\Re}{\mathrm{Re}}
\renewcommand{\hat}{\widehat}
\renewcommand{\tilde}{\widetilde}
\renewcommand{\bar}{\overline}
\renewcommand{\epsilon}{\varepsilon}
% \renewcommand{\span}{\mathrm{span}}
\newcommand{\defi}{\stackrel{\Delta}{\Longleftrightarrow}}
\newcommand{\equi}{\Longleftrightarrow}
\newcommand{\s}{\succsim}
\newcommand{\p}{\precsim}
\newcommand{\join}{\vee}
\newcommand{\meet}{\wedge}
\newcommand{\E}{\mathbbm{E}}
\newcommand{\1}{\mathbbm{1}}

\DeclareMathOperator{\Var}{Var}
\DeclareMathOperator{\Cov}{Cov}
\DeclareMathOperator{\sgn}{sgn}
\DeclareMathOperator{\Card}{Card}
\DeclareMathOperator{\supp}{supp}
\DeclareMathOperator{\Log}{Log}
\DeclareMathOperator{\spn}{span}
\DeclareMathOperator*{\argmin}{argmin}
\DeclareMathOperator*{\argmax}{argmax}

\newcommand{\indep}{\mathop{\perp\!\!\!\!\perp}}

\usepackage{color}
\newcommand{\kcomment}[1]{{\textcolor{blue}{#1}}}
\newcommand{\ocomment}[1]{{\textcolor{red}{#1}}}


\begin{document}
\title{Notes on Mechanism Design}
\author{Kyohei OKUMURA
%{\footnote{E-mail: kyohei.okumura@gmail.com}
%\footnote{UTokyo Econ M2}}
}
\date{\today}
\maketitle

%%%%%%%%%%%%%%%%%%%%%%%%%%%%%%%%%%%%%%%%%%%%%%%%%%%%%%%%%%%
\begin{itemize}
	\item This study notes are mainly based on the lecture note written by Valimaki in 2018.
\end{itemize}

%%%
\section{Single Agent}
\begin{itemize}
	\item One principal v.s. one agent.
	\item $a \in A$: allocation, $\theta \in \Theta$: agent's private info. $\theta \sim F(\theta)$. $u^P(a, \theta)$, $u^A(a, \theta)$.
	\item We often assume quasi-linear payoff functions:
	\begin{itemize}
	\item $a := (x, t)$, $u^P(a, \theta) := v^P(x, \theta) + t$, $u^A(a, \theta) := v^A(x, \theta) - t$.
	\end{itemize}
	\item A mechanism is a pair $M := (\Sigma, \phi)$, where $\Sigma$ is a message space and $\phi: \Sigma \to \Delta(A)$.
	\item Agent's strategy: $\sigma: \Theta \to \Delta(\Sigma)$. Principal commits to a mechanism $M$.
	\item Consider a social choice function $\psi: \Theta \to A$. We want to know whether $\psi$ is implementable (, i.e., achievable in equilibrium,) or not.
	\item As for implementability, we can discuss it focusing only on direct mechanisms, assuming $\Sigma := \Theta$, w.l.o.g. (Revelation principle)
\end{itemize}


%%%
\subsection{Revenue Equivalence}
\begin{itemize}
	\item In \S1.1 and \S1.2, we assume that the parameter space is a closed interval $\Theta := [\underline{\theta}, \bar{\theta}] \subseteq \R$.
\end{itemize}

\subsubsection{Milgrom and Segal (2002), Envelope Theorem}
\begin{itemize}
	\item $\Theta := [\underline{\theta}, \bar{\theta}]$. $f(\cdot, \theta): X \to \R$. $\{f(\cdot, \theta)\}_{\theta \in \Theta}$.
	\item $V(\theta) := \max_{x \in X} f(x, \theta)$. $X^*(\theta) := \argmax_{x \in X} f(x, \theta)$
\end{itemize}
\begin{df}[Selection]
	A function $x^*: \Theta \to X$ is a selection from $X^*$ if $x^*(\theta) \in X^*(\theta)$ for all $\theta \in \Theta$.
\end{df}
\begin{thm}[Milgrom and Segal (2002)]
	Assume the following:
	\begin{itemize}
		\item For any $x \in X$, $f(x, \cdot): \Theta \to \R$ is absolutely continuous on $\Theta$.
		\item For any $x \in X$, $f(x, \cdot): \Theta \to \R$ is differentiable on $\Theta$.
	\end{itemize}
	Then, the following holds:
	\begin{itemize}
		\item $V$ is absolutely continuous.
		\item For any selection $x^*$ from $X^*$, $V(\theta) = V(\underline{\theta}) + \int_{\underline{\theta}}^\theta f_\theta(x^*(s), s) ds$.
	\end{itemize}
\end{thm}
\begin{proof}
	Note that the absolute continuity of $f(x, \theta)$ implies that $f_\theta(x, \theta) \in L^1(\Theta)$ for any $x \in X$.
	\paragraph{(i) $V$ is absolutely continuous.}
	It is sufficient to show that $V$ is Lipschitz continuous. Fix any $\theta', \theta$. Since any integrable function is bounded, for any $x$ there exists $L>0$ s.t. $|f_\theta(x, \theta)| \leq L$ for almost all $\theta \in \Theta$.
	\begin{align*}
		|V(\theta') - V(\theta)|
		&=
		\left|
		\max_{x'} f(x', \theta') - \max_x f(x, \theta)
		\right| \\
		&\leq
		\max_x
		\left|
		f(x, \theta') - f(x, \theta)
		\right| 
		=
		\max_x
		\left|
		\int_{\theta'}^\theta f_\theta(x, s) ds
		\right| \\
		&\leq L \cdot |\theta' - \theta|
	\end{align*}
	
	\paragraph{(ii)}
	Fix any selection $x^*$ from $X^*$. By the result of (i),
	\[
	V(\theta) = V(\underline{\theta}) + \int_{\underline{\theta}}^\theta V'(s) ds
	\]
	Fix any selection $x^*$ and $\theta', \theta$ such that $\theta' > \theta$. By the definition of $V$ and $x^*$,
	\begin{align*}
		V(\theta) &= f(x^*(\theta), \theta) \geq f(x^*(\theta'), \theta) \\
		V(\theta') &= f(x^*(\theta'), \theta') \geq f(x^*(\theta), \theta')
	\end{align*}
	Hence,
	\[
	V(\theta') - V(\theta) \leq f(x^*(\theta'), \theta') - f(x^*(\theta'), \theta).
	\]
	\[
	\frac{V(\theta') - V(\theta)}{\theta' - \theta} \leq \frac{f(x^*(\theta'), \theta') - f(x^*(\theta'), \theta)}{\theta' - \theta}.
	\]
	Similarly,
	\[
	V(\theta) - V(\theta') \leq f(x^*(\theta'), \theta) - f(x^*(\theta'), \theta').
	\]
	\[
	\frac{V(\theta') - V(\theta)}{\theta - \theta'} \geq \frac{f(x^*(\theta'), \theta) - f(x^*(\theta'), \theta')}{\theta - \theta'}.
	\]
	Note that by assumption $f(x, \cdot)$ is differentiable at all $\theta \in \Theta$. Therefore, if $V$ is differentiable at $\theta$, we have $V'(\theta) = f_\theta(x^*(\theta), \theta)$.
\end{proof}


%%%
\subsubsection{RET}
\begin{itemize}
	\item Focus on the agent's utility: $u := u^A$.
	\item $A := \phi(\Theta)$. $V(\theta) := \max_{a \in A} u(a, \theta)$. $A^*(\theta) := \argmax_{a \in A} u(a, \theta)$.
	\item Assume that $u(a, \cdot)$ is absolutely continuous and differentiable on $\Theta$ for all $a \in A$.
	\item By incentive compatibility, $\phi(\theta) \in A^*(\theta)$ for all $\theta \in \Theta$: $\phi$ is a selection from $A^*$.
\end{itemize}

\begin{thm}[Revenue Equivalence Theorem]
	\[
	V(\theta) = V(\underline{\theta}) + \int_{\underline{\theta}}^\theta
	 u_\theta(\phi(s), s) ds
	\]
	In particular, under quasi-linear utility, 
	\[
	V(\theta) = V(\underline{\theta}) + \int_{\underline{\theta}}^\theta
	 v_\theta(x(s), s) ds
	\]
	\[
	t(\theta) = v(x(\theta), \theta) - V(\underline{\theta}) - \int_{\underline{\theta}}^\theta
	 v_\theta(x(s), s) ds
	\]
\end{thm}
\begin{proof}
	Milgrom and Segal. As for quasi-linear cases, the results follow from
	$$
	V(\theta) = v(x(\theta), \theta) - t(\theta)
	$$
\end{proof}

\begin{itemize}
	\item RET states that under any IC mechanism, except for the constant $V(\underline{\theta})$, the transfer from the agent to the principal is uniquely determined once the allocation rule $x$ is fixed.
\end{itemize}


%%%
\subsection{Characterization of IC}
\subsubsection{Monotone Comparative Statics}
This subsection is based on the lecture slides by John K.-H. Quah:

\url{http://www.johnquah.com/lecture-slides.html}

\begin{itemize}
	\item Consider parameterized optimization problems.
	\item We often want to know how optimizers and optimal values change according to the changes in parameters.
	\item comparative statics = Sensitivity analysis
	\item Implicit function theorem: Not only the direction of changes but also the rate of change. Many assumptions are required.
	\item Monotone comparative statics: Only the direction of changes. Fewer assumptions.
	\item $\Theta \subseteq \R$. Two functions $g: \Theta \to \R$ and $f: \Theta \to \R$.
\end{itemize}

\begin{df}[Single Crossing]
	$g$ dominates $f$ by single crossing property (SCP), $g \succsim_{SC} f$, if for all $x'' > x'$,
	\begin{itemize}
		\item $f(x'') - f(x') \geq 0 \implies g(x'') - g(x') \geq 0$
		\item $f(x'') - f(x') > 0 \implies g(x'') - g(x') > 0$
	\end{itemize}
	
	$\{f(\cdot, \theta)\}_{\theta \in \Theta}$ is an SCP family if
	$$
	\forall \theta'' > \theta' ; \ f(\cdot, \theta'') \succsim_{SC} f(\cdot, \theta')
	$$
\end{df}

\begin{df}[Increasing Differences]
	$g$ dominates $f$ by increasing differences, $g \succsim_{IN} f$, if for all $x'' > x'$, $$g(x'') - g(x') \geq f(x'') - f(x').$$
	$\{f(\cdot, \theta)\}_{\theta \in \Theta}$ satisfies increasing differences if
	$$
	\forall \theta'' > \theta' ; \ f(\cdot, \theta'') \succsim_{IN} f(\cdot, \theta')
	$$
\end{df}
\begin{df}[Strictly Increasing Differences]
	$g$ dominates $f$ by strictly increasing differences, $g \succsim_{SID} f$, if for all $x'' > x'$, $$g(x'') - g(x') > f(x'') - f(x').$$
	$\{f(\cdot, \theta)\}_{\theta \in \Theta}$ satisfies strictly increasing differences (SID) if
	$$
	\forall \theta'' > \theta' ; \ f(\cdot, \theta'') \succsim_{SID} f(\cdot, \theta')
	$$
\end{df}

\begin{itemize}
	\item Note that $g \succsim_{IN} f$ implies $g \succsim_{SC} f$.
\end{itemize}

\begin{thm}[Milgrom and Shannon (1994)]
	$X \subseteq \R$. $f,g: X \to \R$.
	\[
	[\forall Y \subseteq X; \ \argmax_{x \in Y} g(x) \geq \argmax_{x \in Y} f(x)
	]
	\equi g \succsim_{SC} f
	\]
	Note that, for $Y,Z \subseteq \R$,
	\[
	Y \geq Z
	\defi  [y \in Y, \ z \in Z
	\implies y \vee x \in Y, \ y \wedge z \in Z.
	]
	\]
\end{thm}
\begin{proof}.
	\paragraph{$\Rightarrow)$}
	We show contrapositive. Suppose that $g \not\succsim_{SC} f$.
	There exist $x'', x'$ such that $x'' > x'$ and at least one of the following holds:
	\begin{equation}
		f(x'') \geq f(x'), g(x'') < g(x') \label{ms1}
	\end{equation}
	or
	\begin{equation}
		f(x'') > f(x'), g(x'') \leq g(x') \label{ms2}
	\end{equation}
	Let $Y := \{x', x''\}$, $G_Y := \argmax_{x \in Y} g(x)$ and $F_Y := \argmax_{x \in Y} f(x)$.
	In case of (\ref{ms1}), $x' \vee x'' \notin G_Y$. In case of (\ref{ms2}), $x' \wedge x'' \notin F_Y$.
	
	\paragraph{$\Leftarrow)$}
	Fix any $Y \subseteq X$ and $x'', x' \in Y$ such that $x' \in G_Y$ and $x'' \in F_Y$. We need to show that $x' \vee x'' \in G_Y$ and $x' \wedge x'' \in F_Y$.
	First, since $x'' \in F_Y$, we have $f(x'') \geq f(x')$. By assumption, $g(x'') \geq g(x')$. Since $x' \in G_Y$, we have $x'' \in G_Y$ and $x' \vee x'' \in G_Y$.
	
	Next, we show $f(x'') = f(x')$. Note that this implies that $x' \wedge x'' \in F_Y$. 
	Suppose toward contradiction that $f(x'') > f(x')$. Then, since $g \succsim_{SC} f$, we have $g(x'') > g(x')$. This contradicts $x' \in G_Y$.
\end{proof}


%%%
\subsubsection{Characterization of IC}

\begin{itemize}
	\item Consider quasi-linear utility cases. Assume that $v(x, \theta)$ is absolutely continuous and differentiable on $\Theta$ for all $x$.
\end{itemize}

\begin{lem}
	Let $V(\theta) := v(x(\theta), \theta) - t(\theta)$.
	If a mechanism $(x,t)$ is IC, then
	\begin{equation}
	V(\theta) = V(\underline{\theta}) + \int_{\underline{\theta}}^\theta
	 v_\theta(x(s), s) ds \tag{LIC}	
	\end{equation}
\end{lem}
\begin{proof}
	RET.
\end{proof}

\begin{lem}
	If a mechanism $(x,t)$ is IC and $\{v(\cdot, \theta)\}_{\theta \in \Theta}$ satisfies SID, then
	\begin{equation}
	\text{$x(\theta)$ is non-decreasing in $\theta$.} \tag{M}
	\end{equation}
\end{lem}
\begin{proof}
	Fix $\theta'', \theta'$ such that $\theta'' > \theta'$. Since $\{v(\cdot, \theta)\}_{\theta \in \Theta}$ satisfies SID, $v(\cdot, \theta'') \succsim_{SID} v(\cdot, \theta')$.
%	By Milgrom-Shannon and IC, 
%	\[
%	x(\theta'')
%	\in \argmax_{x \in x(\Theta)} v(x, \theta'') \geq \argmax_{x \in x(\Theta)} v(x, \theta')
%	\ni x(\theta')
%	\]
	Suppose toward contradiction that $x(\theta'') < x(\theta')$. Since $v(\cdot, \theta'') \succsim_{SID} v(\cdot, \theta')$,
	\[
	v(x(\theta'), \theta'') - v(x(\theta''), \theta'')
	> v(x(\theta'), \theta') - v(x(\theta''), \theta') \geq 0
	\]
	This violates IC. A contradiction.
\end{proof}

\begin{itemize}
	\item The lemmas above shows that, assuming $\{v(\cdot, \theta)\}_{\theta \in \Theta}$ satisfies SID, IC of $(x,t)$ implies (LIC) and (M).
	\item We can show that the converse also holds.
\end{itemize}

\begin{lem}
	Assume that $\{v(\cdot, \theta)\}_{\theta \in \Theta}$ satisfies SID. If the conditions (LIC) and (M) hold, then $(x,t)$ is IC.
\end{lem}
\begin{proof}
	Fix any $\theta, \theta'$. We need to show that $v(x(\theta), \theta) - t(\theta) \geq v(x(\theta'), \theta) - t(\theta')$.
	Note that, by (LIC), we have 
	\[
	t(\theta) = v(x(\theta), \theta) - V(\underline{\theta}) - \int_{\underline{\theta}}^\theta
	 v_\theta(x(s), s) ds
	\]
	
	Then,
	\begin{align*}
		&[v(x(\theta), \theta) - t(\theta)] - [v(x(\theta'), \theta) - t(\theta')] \\
		&=
		[v(x(\theta), \theta) - t(\theta)] - [v(x(\theta'), \theta) + v(x(\theta'), \theta') - v(x(\theta'), \theta') - t(\theta')] \\
		&=
		\int_{\underline{\theta}}^\theta v_\theta(x(s), s) ds
		- \int_{\underline{\theta}}^{\theta'} v_\theta(x(s), s) ds
		- [v(x(\theta'), \theta) - v(x(\theta'), \theta')] \\
		&= \int_{\theta'}^\theta v_\theta(x(s), s) ds
		- \int_{\theta'}^\theta v_\theta(x(s), \theta') ds
		= \int_{\theta'}^\theta [v_\theta(x(s), s) - v_\theta(x(\theta'), s)] ds \geq 0
	\end{align*}
\end{proof}

\begin{thm}[Characterization of IC] \label{ic_char}
	Assume that $\{v(\cdot, \theta)\}_\theta$ satisfies SID. Then, 
	\[
	\text{$(x, t)$ is IC} \equi \text{$x$ is non-decreasing, and $t$ is calculated by (LIC)}
	\]
\end{thm}


%%%
\subsection{General Case: Rochet's Theorem and Cyclical Monotonicity}
\begin{itemize}
	\item Consider quasi-linear utility cases.
	\item Characterize IC mechanisms.
\end{itemize}

\begin{df}[weak monotonicity]
	An allocation rule $x: \Theta \to A$ is weakly monotone if
	\[
	\forall \theta, \theta' ; \ [v(x(\theta), \theta') - v(x(\theta), \theta)]  + [v(x(\theta'), \theta) - v(x(\theta'), \theta')] \leq 0
	\]
\end{df}

\begin{prop}
	If $(x,t)$ is IC, then $x$ is weakly monotone.
\end{prop}

\begin{df}[cyclical monotonicity]
	\[
	S := \{(\theta^1, \cdots, \theta^{k+1}) \mid \forall i \in [k+1]; \theta^i \in \Theta, \ \theta^1 = \theta^{k+1}, \ k \in \Z^+ \}
	\]
	An allocation rule $x$ ie cyclically monotone if
	, for any $(\theta^1, \cdots, \theta^{k+1}) \in S$, 
	\begin{equation}
		\sum_{i=1}^k [v(x^{i}, \theta^{i+1}) - v(x^{i}, \theta^{i})] \leq 0
		\text{ , where $x^i := x(\theta^i)$}
		\tag{CM} \label{cm}
	\end{equation}
\end{df}

\begin{thm}[Rochet (1987)]
	\[
	\exists t; \ (x,t): \text{IC } \equi \text{x is cyclically monotone}.
	\]
\end{thm}

\begin{proof}.
\paragraph{$\Rightarrow)$} Easy.
\paragraph{$\Leftarrow)$}
	Fix $\theta_0 \in \Theta$.
	\[
	S(\theta) := \{(\theta^1, \cdots, \theta^{k+1}) \mid \forall i \in [k+1]; \theta^i \in \Theta, \ \theta^1 = \theta_0, \ \theta^{k+1} = \theta, \ k \in \Z^+ \}
	\]
	\[
	V(\theta) := \sup_{(\theta^1, \cdots, \theta^{k+1}) \in S(\theta)} \sum_{i=1}^k [v(x^{i}, \theta^{i+1}) - v(x^{i}, \theta^{i})]
	\]
	\paragraph{(i) [$V(\theta_0) = 0$.]} By CM, $V(\theta_0) \leq 0$. Considering the case where $k := 1$, we see that $(\theta_0, \theta_0) \in S(\theta_0)$ satisfies $[v(x^1, \theta^2) - v(x^1, \theta^1)] = 0$. Therefore, $V(\theta_0) = 0$.
	
	\paragraph{(ii) [$V(\theta) < \infty$ for all $\theta \in \Theta$.]} Fix any $(\theta^1, \cdots, \theta^{k+1}) \in S(\theta)$.
	\begin{align*}
		0 = V(\theta_0)
		&\geq \sum_{i=1}^k [v(x^{i}, \theta^{i+1}) - v(x^{i}, \theta^{i})] + [v(x^{i+1}, \theta_0) - v(x^{i+1}, \theta^{k+1})] \\
		&= \sum_{i=1}^k [v(x^{i}, \theta^{i+1}) - v(x^{i}, \theta^{i})] + [v(x(\theta), \theta_0) - v(x(\theta), \theta)]
	\end{align*}
	\[
	\therefore 
		\sum_{i=1}^k [v(x^{i}, \theta^{i+1}) - v(x^{i}, \theta^{i})]
		\leq v(x(\theta), \theta) - v(x(\theta), \theta_0)
	\]
	\[
	\therefore 
		V(\theta)
		\leq v(x(\theta), \theta) - v(x(\theta), \theta_0)
	\]
	
	\paragraph{(iii) [Construct the transfer rule]}
	Fix any $\theta, \theta'$. By the same argument as in (ii), we can show that 
	\[
	V(\theta) \geq V(\theta') + v(x(\theta'), \theta) - v(x(\theta'), \theta')
	\]
	Define $t(\theta) := v(x(\theta), \theta) - V(\theta)$. With this $t$, a mechanism $(x,t)$ satisfies IC:
	\[
	v(x(\theta), \theta) - t(\theta) - (v(x(\theta'), \theta) - t(\theta'))
	= V(\theta) - V(\theta') - v(x(\theta'), \theta) + v(x(\theta'), \theta') \geq 0
	\]
\end{proof}



%%%
\subsection{Optimizing over Incentive Compatible Mechanisms}
\begin{ass}[Assumptions for IC characterization]
	In \S1.4, we assume that (1) utility function is quasi linear, (2) $v(x, \theta)$ is absolutely continuous and differentiable on $\Theta$ for all $x$, and (3) $\{v(\cdot, \theta)\}_\theta$ has SID.
\end{ass}
\begin{ass}[Private Values]
	$v^P(x, \theta) \equiv v^P(x)$
\end{ass}
\begin{ass}[Absolutely continuous distribution]
	The distribution function $F$ is absolutely continuous, i.e., 
	there exists $f:\Theta \to \R_+$ s.t. $F(x) := \int_{\underline{\theta}}^x f(s) ds$
\end{ass}

\begin{itemize}
	\item Optimal Mechanism = Revenue Maximizing Mechanism
	\item By Thm. \ref{ic_char}, $(x, t)$ is IC iff $x$ is nondecreasing and $t$ is calculated by Envelope theorem.
\end{itemize}
\begin{align*}
	\text{[Expected Revenue]}
	&=
	\E_\theta \left[
	t(\theta) + v^P(x(\theta))
	\right] \\
	&=
	\E_\theta \left[
	v(x(\theta), \theta) - V(\underline{\theta}) - \int_{\underline{\theta}}^\theta
	 v_\theta(x(s), s) ds + v^P(x(\theta))
	\right] \\
	&= \E_\theta \left[
	S(x(\theta), \theta) - V(\underline{\theta}) - \int_{\underline{\theta}}^\theta
	 v_\theta(x(s), s) ds
	\right]
\end{align*}
\begin{align*}
	\E_\theta \left[
	\int_{\underline{\theta}}^\theta
	 v_\theta(x(s), s) ds
	\right]
	&=
	\int_{\underline{\theta}}^{\bar{\theta}}
	\int_{\underline{\theta}}^\theta
	 v_\theta(x(s), s) ds
	d F(\theta) \\
	&= 
	\int_{\underline{\theta}}^{\bar{\theta}}
	\int_{s}^{\bar{\theta}}
	 v_\theta(x(s), s) 
	dF(\theta) ds \\
	&= 
	\int_{\underline{\theta}}^{\bar{\theta}}
	 (1 - F(s)) v_\theta(x(s), s)
	 ds \\
	&= 
	\int_{\underline{\theta}}^{\bar{\theta}}
	 (1 - F(\theta)) v_\theta(x(\theta), \theta)
	 d\theta \\
	&= 
	\int_{\underline{\theta}}^{\bar{\theta}}
	 v_\theta(x(\theta), \theta) \frac{1 - F(\theta)}{f(\theta)}
	 dF(\theta) \\
	&=
	\E_\theta \left[
	 v_\theta(x(\theta), \theta) \frac{1 - F(\theta)}{f(\theta)}
	\right]
\end{align*}
\[
\therefore \quad
\text{[Expected Revenue]}
=
\E_\theta \left[
	S(x(\theta), \theta) - V(\underline{\theta})
	- v_\theta(x(\theta), \theta) \frac{1 - F(\theta)}{f(\theta)}
	\right]
\]
\begin{itemize}
	\item The principal solves the following revenue maximization problem:
\end{itemize}
\[
\max_{x(\cdot), V(\underline{\theta})}
\E_\theta \left[
	S(x(\theta), \theta) - V(\underline{\theta})
	- v_\theta(x(\theta), \theta) \frac{1 - F(\theta)}{f(\theta)}
	\right]
\text{ s.t. }
x(\cdot): \text{increasing.}
\]

\begin{itemize}
	\item It is optimal to set $V(\underline{\theta}) := 0$, assuming that the outside option value is zero. Then, the problem can be reduced to
\end{itemize}
\[
\max_{x(\cdot)}
\E_\theta \left[
	\underbrace{S(x(\theta), \theta)
	- v_\theta(x(\theta), \theta) \frac{1 - F(\theta)}{f(\theta)}
	}_{(\star)}
	\right]
\text{ s.t. }
x(\cdot): \text{nondecreasing.}
\]
\begin{itemize}
	\item If $v^P(\theta) \equiv 0$,
	$(\star) = v(x(\theta), \theta)
	- v_\theta(x(\theta), \theta) \frac{1 - F(\theta)}{f(\theta)}
	=:$ [the virtual valuation of the bidder.]
\end{itemize}

\subsection*{Finding a Solution}
\begin{itemize}
	\item Just ignoring the monotonicity of $x$ and solve the relaxed problem. Fix $\theta \in \Theta$, and solve maximization problem for each $\theta$.
	\item \ocomment{[Is the argument below valid in case $x$ is not $\mC^1$ on $\Theta$?]}
\end{itemize}
\begin{ass}
	Assume that $v$ is linear in $\theta$.
\end{ass}
\begin{ass}
	Assume the interior solution. (?)
\end{ass}
\[
\max_x S(x, \theta) - v_\theta(x, \theta) \frac{1 - F(\theta)}{f(\theta)}
\]
\begin{equation}
	S_{x}(x, \theta) - v_{\theta x}(x, \theta) \frac{1 - F(\theta)}{f(\theta)} = 0  \tag{FOC}
\end{equation}
\begin{align*}
	&\underbrace{
	\left(
	\underbrace{
	S_{x\theta}(x, \theta)
	}_{= v_{x\theta}(x, \theta) \ (\because \ PV)}
	- v_{\theta x}(x, \theta) \frac{d}{d\theta}\frac{1 - F(\theta)}{f(\theta)}
	- \underbrace{
	v_{\theta x \theta}(x, \theta)}_{=0}
	\frac{1 - F(\theta)}{f(\theta)}
	\right)
	}_{= v_{x\theta}(x, \theta) \left(1 - \frac{d}{d\theta}\frac{1 - F(\theta)}{f(\theta)} \right)}
	d\theta \\
	&+
	\left(
	\underbrace{
	S_{xx}(x, \theta)
	- v_{\theta xx}(x, \theta) \frac{1 - F(\theta)}{f(\theta)}
	}_{\leq 0 \quad (\because \ SOC)}
	\right) dx
	= 0 \\
	&\equi
	\left( v_{x\theta}(x, \theta)  - v_{x\theta}(x, \theta) \frac{d}{d\theta}\frac{1 - F(\theta)}{f(\theta)} \right) d\theta + (\cdots)dx = 0 \\
	&\therefore \quad \sgn \left( \frac{dx}{d\theta} \right) = \sgn\left( v_{x\theta}(x, \theta)  - v_{x\theta}(x, \theta) \frac{d}{d\theta}\frac{1 - F(\theta)}{f(\theta)} \right) 
\end{align*}

\begin{itemize}
	\item Note that since $v$ has SID, $v_{x \theta} \geq 0$.
	\item The following condition (MHR: monotone hazard rate condition) is sufficient in order for $x$ to be nondecreasing:
\end{itemize}
\[
\frac{d}{d\theta}\frac{1 - F(\theta)}{f(\theta)} \leq 0
\equi \frac{d}{d\theta}\frac{f(\theta)}{1 - F(\theta)} \geq 0
\]


%%%
\newpage
\section{Many Agents}








\end{document}