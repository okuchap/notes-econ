\documentclass[11pt,a4paper,dvipdfmx]{article}
%\documentclass[autodetect-engine,dvipdfmx-if-dvi,ja=standard]{bxjsarticle}

\usepackage{ascmac}
\usepackage[utf8]{inputenc}
\usepackage{lmodern}
\usepackage[T1]{fontenc}
\usepackage[noBBpl]{mathpazo}
%\linespread{1.05}
\usepackage{mathtools, amsmath, amssymb, amsthm}
\usepackage{latexsym}
\usepackage{amsfonts}
\usepackage{braket}
%\usepackage{amssymb}
\usepackage{url}
\usepackage{cases}
\usepackage{bbm}
\usepackage[all]{xy}

%% citation
\usepackage[longnamesfirst]{natbib}

%
\theoremstyle{plain}
\newtheorem{thm}{Thm.}[section]
\newtheorem{lem}{Lem.}[section]
\newtheorem{cor}{Cor.}[section]
\newtheorem{prop}{Prop.}[section]
\newtheorem{df}{Def.}[section]
\newtheorem{eg}{e.g.}[section]
\newtheorem{rem}{Rem.}[section]
\newtheorem{ass}{Ass.}[section]
%

\usepackage{listings}
\lstset{%
language={python},%
basicstyle={\ttfamily\footnotesize},%ソースコードの文字を小さくする
frame={single},
commentstyle={\footnotesize\itshape},%コメントアウトの文字を小さくする
breaklines=true,%行が長くなったときの改行。trueの場合は改行する。
numbers=left,%行番号を左に書く。消す場合はnone。
xrightmargin=3zw,%左の空白の大きさ
xleftmargin=3zw,%右の空白の大きさ
stepnumber=1,%行番号を1から始める場合こうする(たぶん)
numbersep=1zw,%行番号と本文の間隔。
}

%\usepackage[dvipdfmx]{graphicx}
%% color packageとdvipdfmxは相性が悪いらしい
%% https://qiita.com/zr_tex8r/items/442b75b452b11bee8049
\usepackage{graphicx}


\usepackage[left=2cm,right=2cm,top=2cm,bottom=2cm]{geometry} %This changes the margins.
\usepackage{float}
%\author{Kyohei Okumura}
\global\long\def\T#1{#1^{\top}}

\newcommand{\id}{\textnormal{id}}
\newcommand{\R}{\mathbb{R}}
\newcommand{\N}{\mathbb{N}}
\newcommand{\Q}{\mathbb{Q}}
\newcommand{\Z}{\mathbb{Z}}
\newcommand{\C}{\mathbb{C}}
\newcommand{\mF}{\mathcal{F}}
\newcommand{\mG}{\mathcal{G}}
\newcommand{\mA}{\mathcal{A}}
\newcommand{\mB}{\mathcal{B}}
\newcommand{\mC}{\mathcal{C}}
\newcommand{\mD}{\mathcal{D}}
\newcommand{\mE}{\mathcal{E}}
\newcommand{\mL}{\mathcal{L}}
\newcommand{\mM}{\mathcal{M}}
\newcommand{\mO}{\mathcal{O}}
\newcommand{\mP}{\mathcal{P}}
\newcommand{\mS}{\mathcal{S}}
\newcommand{\mT}{\mathcal{T}}
\newcommand{\mV}{\mathcal{V}}
\newcommand{\mX}{\mathcal{X}}
\renewcommand{\Re}{\mathrm{Re}}
\renewcommand{\hat}{\widehat}
\renewcommand{\tilde}{\widetilde}
\renewcommand{\bar}{\overline}
\renewcommand{\epsilon}{\varepsilon}
% \renewcommand{\span}{\mathrm{span}}
\newcommand{\defi}{\stackrel{\Delta}{\Longleftrightarrow}}
\newcommand{\equi}{\Longleftrightarrow}
\newcommand{\s}{\succsim}
\newcommand{\p}{\precsim}
\newcommand{\join}{\vee}
\newcommand{\meet}{\wedge}
\newcommand{\E}{\mathbbm{E}}
%\newcommand{\1}{\mathbbm{1}}
\newcommand{\indep}{\mathop{\perp\!\!\!\!\perp}}

\DeclareMathOperator{\Var}{Var}
\DeclareMathOperator{\Cov}{Cov}
\DeclareMathOperator{\sgn}{sgn}
\DeclareMathOperator{\Card}{Card}
\DeclareMathOperator{\supp}{supp}
\DeclareMathOperator{\Log}{Log}
\DeclareMathOperator{\spn}{span}
\DeclareMathOperator*{\argmin}{argmin}
\DeclareMathOperator*{\argmax}{argmax}

\usepackage{color}
\newcommand{\kcomment}[1]{{\textcolor{blue}{#1}}}
\newcommand{\ocomment}[1]{{\textcolor{red}{#1}}}


\begin{document}
\title{Spiegler (2017, REStud), Data Monkeys, Proposition 2
}
\author{Kyohei Okumura{\footnote{E-mail: kyohei.okumura@gmail.com}
}}
\date{\today}
\maketitle

%%%%%%%%%%%%%%%%%%%%%%%%%%%%%%%%%%%%%%%%%%%%%%%%%%%%%%%%%%%

\begin{itemize}
	\item The equivalence class of a DAG $R$ is denoted as $[R]$
	\item The v-structure of a DAG $R$ is denoted as $v(R) := \{(i,j,k) \mid i \to j, j \to k, i \nrightarrow j, j \nrightarrow i \}$.
\end{itemize}

\begin{prop}[Spiegler(2017, REStud) Prop.2, Spiegler(2016, QJE) Prop.7] \label{prop_sp2017} 
	Let $R$ be a DAG and let $C \subseteq N$.
	\[
	[\forall p \in \Delta(X) \forall x; p_R(x_C) = p(x_C)]
	\equi
	[\exists Q \in [R]; C \text{ is an ancestral clique in } Q].
	\]
	\ocomment{[2018/07/16: $\Leftarrow$ is correct; $\Rightarrow$ is not sure.]}
\end{prop}

\begin{proof}
	First, note that for any DAG $R$, the following holds:
	\begin{align}
		p_R(x_C) &= \sum_{x'_{N-C}} p_R(x_C, x'_{N-C}) \nonumber \\
		&=  \sum_{x_{N-C}}
		\prod_{i \in C} p(x_i \mid x_{R(i) \cap C}, x'_{R(i) - C})
		\prod_{i \notin C} p(x'_i \mid x_{R(i) \cap C}, x'_{R(i) - C}) \label{eq1}
	\end{align}
	
	\paragraph{$\Leftarrow)$}
	Fix $C$ such that $C$ is an ancestral clique in some $Q \in [R]$. Note that $R(i) - C = \emptyset$ for all $i \in C$. Then,
	\[
	\prod_{i \in C} p(x_i \mid x_{R(i) \cap C}, x'_{R(i) - C})
	= \prod_{i \in C} p(x_i \mid x_{R(i) \cap C})
	= p(x_C) \ (\because \text{topological sort})
	\]
	Hence, by (\ref{eq1}),
	\[
	p_R(x_C) = p_Q(x_C) = p(x_C)
	\underbrace{\sum_{x_{N-C}}
		\prod_{i \notin C} p(x'_i \mid x_{R(i) \cap C}, x'_{R(i) - C})}_{1} = p(x_C).
	\]
	
	\begin{screen}
	\begin{eg}
		For example, consider the following DAG:
		\[
		\xymatrix{
			1 \ar[r] & 2 \ar[rd] & 4 \\
			& & 3
		}
		\]
		Let $C:=\{1,2\}$. Then,
		\[
		p_R(x_1, x_2) = 
		\sum_{x'_3, x'_4} p_R(x_1, x_2, x'_3, x'_4)
		= p(x_1, x_2) \sum_{x'_3, x'_4} p(x'_4) p(x'_3 \mid x_2)
		= p(x_1, x_2)
		\]
	\end{eg}
	\end{screen}
	
	\paragraph{$\Rightarrow)$}
	\ocomment{[We need to make some fix in this direction.]}
	
	We show contrapositive: we show the following:
	\[
	[\forall Q \in [R]; \ C \text{ is not an ancestral clique in } Q]
	\implies
	[\exists p  \exists x; \ p_R(x_C) = p(x_C)]
	\]
	Fix any $Q \in [R]$ that $C$ is not an ancestral clique in $Q$. We divide the proof into two cases:
	\paragraph{Case (i): In case $C$ is not a clique in $Q$.}
	OK.
%	In this case, $C$ is not a clique in any $R' \in [R]$. There must be two distinct nodes $i_0, i_1 \in C$ such that $(i_0, i_1) \notin Q$ and $(i_1, i_0) \notin Q$. Consider $p \in \Delta(X)$ such that for every $i \in C \setminus \{i_0, i_1\}$, $x_i$ is independently distributed, whereas $x_{i_0}$ and $x_{i_1}$ are mutually correlated. Then,
%	\[
%	\prod_{i \in C} p(x_i \mid x_{R(i) \cap C}, x'_{R(i) - C})
%	= \prod_{i \in C}p(x_i) \ (\because \text{there is no edge b/w $i_0$ and $i_1$})
%	\]
%	\[
%	\prod_{i \notin C} p(x'_i \mid x_{R(i) \cap C}, x'_{R(i) - C})
%	= \prod_{i \notin C}p(x'_i)
%	\]
%	\[
%	p_R(x_C) = (\text{\ref{eq1}}) =  \prod_{i \in C}p(x_i) \sum_{i \notin C} \prod_{i \notin C}p(x'_i) = \prod_{i \in C}p(x_i)
%	\]
%	However, 
%	\[
%	p(x_C) = p(x_{i_0}) p(x_{i_1} \mid x_{i_0}) \prod_{i \in C \setminus \{i_0, i_1\}} p(x_i) 
%	\]
%	Therefore, for some $p$, $p_R(x_C) \neq p(x_C)$.
	
	%%%
	\paragraph{Case (ii): In case $C$ is a clique, but not an ancestral clique.}
	In the original proof in Spiegler(2017), there is a lemma like the following (pp.1838, Case 2, the first paragraph), but the lemma is wrong:
	
	\begin{screen}
	\begin{lem} \label{wrong_lem}
	Let $R$ be a DAG. Assume the following two:
		\begin{enumerate}
			\item $\forall j \in C$; $j$ has no unmarried parents in $R$.
			\item $\forall i \notin C$; if there is a directed path from $i$ to some node $j \in C$ in $R$, then $i$ has no unmarried parents in $R$.
		\end{enumerate}
		Transform $R$ into another DAG $R'$ by inverting every link along every such path; $R$ and $R'$ has the same v-structure.
	\end{lem}
	\end{screen}
	
	\begin{screen}
	\begin{eg}[Counter example for Lem.\ref{wrong_lem}]
		Let $R$ be the graph below:
		\[
		\xymatrix{
			  & j_1 \ar[r] & j_2 \ar[rd]  &  \\
			  & & & c_j \ar[dd] \ar[llddd] \ar[lddd] \\
			i \ar[ruu] \ar[rruu] \ar[rrru] \ar[rdd] & & & \\
			& & & c_k\\
			& k_1 \ar[r] & k_2 \ar[ru] & 
		}
		\]
		Let $C := \{c_j, c_k\}$. Note that for all $k \in N \setminus C$ such that $k$ has a path to some $c \in C$, $k$ has no unmarried parents. $R'$ is as follows:
		\[
		\xymatrix{
			  & j_1 \ar[ldd] & j_2 \ar[l] \ar[lldd] &  \\
			  & & & c_j \ar[dd] \ar[llddd] \ar[lddd] \ar[llld] \ar[lu] \\
			i  & & & \\
			& & & c_k \ar[ld] \\
			& k_1 \ar[luu] & k_2 \ar[l] & 
		}
		\]
		Though $v(R) = \emptyset$, we have $v(R') = \{(j_1, i, k_1), (j_1, i, c_j), (j_2, i, k_1)\}$. Therefore, Lem.\ref{wrong_lem} does not hold.
	\end{eg}
	\end{screen}	
\end{proof}



%%%
\newpage




\end{document}