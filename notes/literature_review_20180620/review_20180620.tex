\documentclass[11pt,a4paper,dvipdfmx]{article}
%\documentclass[autodetect-engine,dvipdfmx-if-dvi,ja=standard]{bxjsarticle}

\usepackage{ascmac}
\usepackage[utf8]{inputenc}
\usepackage{lmodern}
\usepackage[T1]{fontenc}
\usepackage[noBBpl]{mathpazo}
%\linespread{1.05}
\usepackage{mathtools, amsmath, amssymb, amsthm}
\usepackage{amsfonts}
\usepackage{braket}
%\usepackage{amssymb}
\usepackage{url}
\usepackage{bbm}

%% citation
\usepackage[longnamesfirst]{natbib}

%
%\theoremstyle{plain}
%\newtheorem{thm}{Thm.}[section]
%\newtheorem{lem}{Lem.}[section]
%\newtheorem{cor}{Cor.}[section]
%\newtheorem{prop}{Prop.}[section]
%\newtheorem{df}{Def.}[section]
%\newtheorem{eg}{e.g.}[section]
%\newtheorem{rem}{Rem.}[section]
%\theoremstyle{plain}
\newtheorem{thm}{Thm.}
\newtheorem{lem}{Lem.}
\newtheorem{cor}{Cor.}
\newtheorem{prop}{Prop.}
\newtheorem{df}{Def.}
\newtheorem{eg}{e.g.}
\newtheorem{rem}{Rem.}
%

\usepackage{listings}
\lstset{%
language={python},%
basicstyle={\ttfamily\footnotesize},%ソースコードの文字を小さくする
frame={single},
commentstyle={\footnotesize\itshape},%コメントアウトの文字を小さくする
breaklines=true,%行が長くなったときの改行。trueの場合は改行する。
numbers=left,%行番号を左に書く。消す場合はnone。
xrightmargin=3zw,%左の空白の大きさ
xleftmargin=3zw,%右の空白の大きさ
stepnumber=1,%行番号を1から始める場合こうする(たぶん)
numbersep=1zw,%行番号と本文の間隔。
}

%\usepackage[dvipdfmx]{graphicx}
%% color packageとdvipdfmxは相性が悪いらしい
%% https://qiita.com/zr_tex8r/items/442b75b452b11bee8049
\usepackage{graphicx}


\usepackage[left=2cm,right=2cm,top=2cm,bottom=2cm]{geometry} %This changes the margins.
\usepackage{float}
%\author{Kyohei Okumura}
\global\long\def\T#1{#1^{\top}}

\newcommand{\id}{\textnormal{id}}
\newcommand{\R}{\mathbb{R}}
\newcommand{\N}{\mathbb{N}}
\newcommand{\Q}{\mathbb{Q}}
\newcommand{\Z}{\mathbb{Z}}
\newcommand{\C}{\mathbb{C}}
\newcommand{\mF}{\mathcal{F}}
\newcommand{\mG}{\mathcal{G}}
\newcommand{\mA}{\mathcal{A}}
\newcommand{\mB}{\mathcal{B}}
\newcommand{\mC}{\mathcal{C}}
\newcommand{\mD}{\mathcal{D}}
\newcommand{\mE}{\mathcal{E}}
\newcommand{\mL}{\mathcal{L}}
\newcommand{\mM}{\mathcal{M}}
\newcommand{\mN}{\mathcal{N}}
\newcommand{\mO}{\mathcal{O}}
\newcommand{\mP}{\mathcal{P}}
\newcommand{\mS}{\mathcal{S}}
\newcommand{\mT}{\mathcal{T}}
\newcommand{\mV}{\mathcal{V}}
\newcommand{\mW}{\mathcal{W}}
\newcommand{\mX}{\mathcal{X}}
\newcommand{\mZ}{\mathcal{Z}}
\renewcommand{\Re}{\mathrm{Re}}
\renewcommand{\hat}{\widehat}
\renewcommand{\tilde}{\widetilde}
\renewcommand{\bar}{\overline}
\renewcommand{\epsilon}{\varepsilon}
% \renewcommand{\span}{\mathrm{span}}
\newcommand{\defi}{\stackrel{\Delta}{\Longleftrightarrow}}
\newcommand{\equi}{\Longleftrightarrow}
\newcommand{\s}{\succsim}
\newcommand{\p}{\precsim}
\newcommand{\join}{\vee}
\newcommand{\meet}{\wedge}
%\newcommand{\1}{\mbox{1}\hspace{-0.25em}\mbox{l}}
\newcommand{\1}{\mathbbm{1}}
\newcommand{\E}{\mathbbm{E}}

\DeclareMathOperator{\Var}{Var}
\DeclareMathOperator{\Cov}{Cov}
\DeclareMathOperator{\sgn}{sgn}
\DeclareMathOperator{\Card}{Card}
\DeclareMathOperator{\supp}{supp}
\DeclareMathOperator{\Log}{Log}
\DeclareMathOperator{\spn}{span}

\newcommand{\indep}{\mathop{\perp\!\!\!\!\perp}}

\usepackage{color}
\newcommand{\kcomment}[1]{{\textcolor{blue}{#1}}}
\newcommand{\ocomment}[1]{{\textcolor{red}{#1}}}

%%%%%%%%%%%%%%%%%%%%%%%%%%%%%%%%%%%%%%%%%%%%%%%

\begin{document}
\title{サーチ活動}
\author{Kyohei Okumura{\footnote{E-mail: kyohei.okumura@gmail.com}
}}
\date{\today}
\maketitle

\section{Kremer, Mansour and Perry (2014, JPE) \\ ``Implementing the Wisdom of the Crowd''}

\begin{itemize}
	\item Recommendation Mechanism
	\begin{itemize}
		\item Trip Advisor・AirBNB・食べログ・car navigation system・(Amazon) etc.
	\end{itemize}
	\item Information Design, Bandit with Incentives
\end{itemize}
\subsection{Model}
\begin{itemize}
	\item One central planner + $T$ agents
	\item two arms: $A := \{a_1, a_2\}$
	\item The arm $a_i$ has its reward $R_i$.
	\begin{itemize}
		\item Before exploration, no one knows the exact value of $R_i$; once someone explores $a_i$, CP knows its value. (deterministic reward)
		\item $R_i \sim \pi_i, \ \E_{\pi_i}[R_i] = \mu_1, \ \mu_1 \geq \mu_2$.
		\item $\pi$: the joint distribution, common knowledge.
	\end{itemize}
	\item Each agent $t$ knows his position in a line.
	\item CP knows entire history: CP's recommendation, agents' choices, and their rewards.
	\item CP commits to a message policy.
	\item Each period, CP sends a message $m^t$ to agent $t$, agent $t$ decides the arm to choose, and agent $t$ receives his reward $R^t$.
	\item Agent $t$ chooses the arm $a_i^* \in \arg \max_{a_i} \{\E[a_i \mid m^t]\}$
	\item CP wants to maximize the social welfare:
		$$\E\left[\frac{1}{T}R^t \right]$$
	\item What is the best policy?
\end{itemize}

\subsection{Example}
\begin{itemize}
	\item $R_1 \sim U[-1,5], \ R_2 \sim U[-5,5]$. $\mu_1 = 2, \ \mu_2 = 0$.
	\item Each agent arrives sequentially: agent $t$ arrives in period $t$.
	\item Suppose for simplicity that CP wants to explore both arms as soon as possible.
	\item NB: Agent 1 always chooses $a_1$.
\end{itemize}

\subsubsection{Full transparency}
\begin{itemize}
	\item If CP discloses all information, agent 2 explores $a_2$ only if $R_1 \leq 0$.
\end{itemize}

\subsubsection{Threshold policy}
\begin{itemize}
	\item Consider the following recommendation policy:
	$$
	m^2 := \begin{cases}
		a_2 & (R_1 \leq 1) \\
		a_1 & (R_1 > 1)
	\end{cases}
	$$
	$$\E[R_1 \mid m^2 = a_2] = 0 \leq m_2$$
	\item Whenever CP recommends $a_2$ to agent 2, agent 2 follows the recommendation.
	\item Pr(exploration) $\uparrow$
	\item As for agent $3$ we can construct the threshold policy:
	$$
	m^3 := \begin{cases}
		a_2 & ([R_1 \leq 1, R_2 > R_1] \text{ or } [R_1 \in (1, 1+x]]) \\
		a_1 & \text{o.w.}
	\end{cases}
	$$
	\item In case $R_1 \in [-1,1]$, agent 2 explores $a_2$.
	\item In case $R_1 \in (1, 1+x]$, agent 3 explores $a_3$.
	\item $\dots$
\end{itemize}

\subsection{Results}
\begin{itemize}
	\item We can focus on the specific class of policies: \textit{recommendation policy}
	\begin{itemize}
		\item Revelation-principle-like argument
		\item Myerson (1986), \textbf{Sugaya and Wolitzky (2017)}
	\end{itemize}
	\item The optimal policy takes the form of threshold policy.
	\item Extension: Imperfect information about location, \textbf{stochastic reward}
\end{itemize}

\subsection{Related Literature}
\subsubsection{Frazier et al. (2014, EC'14), ``Incentivizing Exploration''}
\begin{itemize}
	\item Use monetary transfer to resolve exploration-exploitation tradeoff.
\end{itemize}
\subsubsection{Bergemann and Valimaki (1996), ``Learning and Strategic Pricing''}
\begin{itemize}
	\item One consumer v.s. Many firms, firms = arms
	\item Arms themselves can set strategic prices for begin pulled.
\end{itemize}
\subsubsection{Abraham et al.(2013), ``Adaptive Crowdsourcing Algorithms for the Bandit Survey Problem''}
\begin{itemize}
	\item crowd-sourcing of tasks
	\item arms = agents
	\item learn the quality of agents' work from observing them while providing enough incentives for them to work
\end{itemize}

\subsubsection{Mansour et al. (2018, EC'15), ``Bayesian Incentive-Compatible Bandit Exploration''}
\begin{itemize}
	\item Generalization of KMP (2014)
	\item $K$ arms
	\item stochastic rewards
	\item detail-free
	\item near-optimal performance(?)
\end{itemize}

\subsubsection{Bimplins et al. (Management Science,  forthcoming), ``Crowdsourcing Exploration''}
\begin{itemize}
	\item Generalization of KMP
	\item two arms
	\item stochastic rewards
	\item less-than-fully informative policy
	\item information design + monetary transfer
	\item optimal (not first-best) policy can be obtained by solving large-scale LP (intractabel)
	\item also propose a heuristic solution that is near-optimal in numerical experiments
\end{itemize}


\subsection{Possible Extension}
\begin{itemize}
	\item Bounded rationality?: It is not realistic to assume that agents are perfectly Bayesian-rational.
	\begin{itemize}
		\item ignorance of selection-bias?
		\item Agents may not recognize the possibility that some firms incentivize reviewers by monetary payoffs.
	\end{itemize}
\end{itemize}

\section{Others}
\subsection{Bimpikins et al. (Operations Research, forthcoming) \\ ``Designing Dynamic Contests''}

\begin{itemize}
	\item contests, learning, dynamic competition, information design
	\item Innovation contests by firms.
	\begin{itemize}
		\item outsource innovation to the crowd
		\item winners are awarded by prize
	\end{itemize}
	\item How to best design a contest?
	\begin{itemize}
		\item What is the best information disclosure policy?: \\
		Whether and when should the contest designer disclose the information regarding the competitors' partial progress?
		\item What is the role of intermediate awards?
	\end{itemize}
	\item 成功するかどうかわからない.
	\item 最初に成功した人のみが報酬をもらう.
	\item Tradeoff: encouragement effect v.s. competition effect
	\begin{itemize}
		\item 他人が成功していることがわかる $\rightarrow$ 成功しやすいプロジェクトなので自分も参加したい.
		\item 他人が成功していることがわかる $\rightarrow$ 自分が遅れていると,今から参入しても勝てなさそうなので参加したくない.
	\end{itemize}
\end{itemize}

\subsection{Weyl and Zhang (2018), ``Depreciating Licenses''}
\begin{itemize}
	\item simple DMD
\end{itemize}

\subsection{Kleinberg, Waggoner and Weyl (2016), ``Descending Price Optimally Coordinates Search''}
\begin{itemize}
	\item information acquisition cost
	\item Descending auctions are ``better'' than ascending auctions? 
\end{itemize}
\end{document}